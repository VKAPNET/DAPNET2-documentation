\chapter{Setup and Installation}


\section{Accessible ports from HAMNET}

\section{Unipager}

\section{DAPNET-Proxy}

\section{DAPNET Core}

\section{Special issues for Core running in Hamcloud}

\subsection{Accessible ports from internet}
To offer the endpoints to internet-based transmitters and users, the following port have to be accessible:

\begin{tabular}{r|r|l}
Type & Port & Application\\
\hline
TCP & 80 & HTTP Webinterface and Websocket\\
TCP & 443 & HTTPS Webinterface  and Websocket\\
TCP & 4369 & RabbitMQ peer discovery\\
TCP & 5672 & RabbitMQ Client connection\\
TCP & 5671 & RabbitMQ-TLS Client connection\\
TCP & 25672 & RabbitMQ Federation Internode Connection\\
TCP & 1883 & MQTT Third Party clients\\
TCP & 8883 & MQTT-TLS Third Party clients\\
\end{tabular}

\todo{check if TCP/25672 is correct for federation}

\subsection{Load balancing and high availability}
\textbf{Internet-based}\\
To offer load balancing and high availability, the internet-based DNS record \textit{hampager.de} would use DNS round-robin with the static internet IPs of the Hamcloud instances.

\textbf{HAMNET/Hamcloud-based}\\
The Hamcloud instances would offer an anycast IP to for transmitter and user connections. There is a special subnet of 44.0.0.0/8 IPs designated for this anycast approach. Besides, the Hamcloud DAPNET instances will have unicast IPs for administration and their inter-node-synchronization.
To connect other nodes besides from the three hamcloud instances, the endpoint to be attached will also be distributed via anycast for maximal fail-over capability.
\todo{rework with content of discussion from 2.8.2018 on network structure}




